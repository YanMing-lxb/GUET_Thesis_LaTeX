% !Mode:: "TeX:UTF-8"

\chapter{多级嵌入式散热装置系统级性能分析及优化}

\begin{table}[h]
    \caption{计算$2m\times 2m$理想导体平板时域感应电流采用的三种存储方式的存储量比较。}
    \begin{tabular}{cccc}
        \toprule
        \multirow{2}{*}{时间步长} & \multicolumn{3}{c}{存储方式} \\
        \cmidrule{2-4}
        & 非压缩存储方式 & 完全压缩存储方式 & 基权函数压缩存储方式 \\
        \midrule
        0.4ns & 5.59 MB & 6.78 MB & 6.78 MB\\
        0.5ns & 10.17 MB & 5.58 MB & 5.58 MB \\
        0.6ns & 8.38MB & 4.98 MB & 4.98 MB \\
        \bottomrule
        \end{tabular}
    \label{tablea}
    \end{table}

\begin{theorem}
如果时域混合场积分方程是时域电场积分方程与时域磁场积分方程的线性组合。
\end{theorem}
\begin{proof}
由于时域混合场积分方程是时域电场积分方程与时域磁场积分方程的线性组合,因此时域混合场积分方程时间步进算法的阻抗矩阵特征与时域电场积分方程时间步进算法的阻抗矩阵特征相同。
\end{proof}
\begin{corollary}
时域积分方程方法的研究近几年发展迅速,在本文研究工作的基础上,仍有以下方向值得进一步研究。
\end{corollary}
\begin{lemma}
因此时域混合场积分方程时间步进算法的阻抗矩阵特征与时域电场积分方程时间步进算法的阻抗矩阵特征相同。
\end{lemma}