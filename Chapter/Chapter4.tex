% !Mode:: "TeX:UTF-8"

\chapter{基于嵌入式散热模块的微通道多目标结构优化}\label{ch:4}

\section{概述}
本章在\cref{ch:3}完成基……。

\section{列表示例}\label{sec:enumerate}

\subsection{普通列表示例}
\begin{enumerate}
    \item 在基板内部进行微通道散热以缩短传热路径,见\cref{fig:LTCC-Microchannels};
    \item 在基板内嵌入散热模块减少整体热阻,提高热传导效率,见\cref{fig:Embedded-cooling-module};
    \item 在嵌入式散热模块上制作针鳍或肋增强对流传热,以进一步减小热阻,见\cref{fig:Rib-pin-fin}。
\end{enumerate}

\subsection{标号为阿拉伯数字的列表}

\begin{enumerate}[label =(\arabic*)]

    \item 基于嵌入式散热模块的微通道流动与传热性能研究。
          将三种带有嵌入式散热模块的微通道:带有针鳍……
          最终选用MC-RPF作为核心散热结构;
    \item 分析几何参数对带有针鳍-肋嵌入式散热模块微通道流动与传热的影响。
          主要研究……;
    \item 对采用针鳍-肋嵌入式散热模块的微通道进行多目标优化。
          采用响应面分析法(Response Surface Methodology,RSM)与……;
    \item 基于MC-RPF的多热源散热结构设计分析。
          为解决在多热源应……;
    \item 基于MC-RPF的多热源散热结构压降优化。
          以压降损失相关理论为指导依据,……。

\end{enumerate}

\subsection{自定义列表标号}
\noindent NSGA-Ⅱ具体操作步骤如下:
\begin{enumerate}[leftmargin = 6em, labelsep = 0em]
    \item[步骤一、] 随机生成初始化种群,设置代数$Gen = 0$;
    \item[步骤二、] 判断是否生成第一代种群,如已生成则令其代数$Gen = 2$,否则进行快速非支配排序、选择、SBX、PM生成第一代子群,并设置代数$Gen = 2$;
    \item[步骤三、] 将父代与子代的种群进行合并形成新的父代种群;
    \item[步骤四、] 判断是否生成新的父代种群,如果未生成则进行快速非支配排序、拥挤度计算、精英策略选择操作以生成新的父代;
    \item[步骤五、] 对新生成的父代进行选择、SBX、PM操作生成新子群;
    \item[步骤六、] 判断当前代数是否小于设置的最大代数,若小于设置的最大代数则返回步骤三进行循环,否则,NSGA-Ⅱ结束运行。
\end{enumerate}

\section{本章小节}