%
%  =======================================================================
%  ····Y88b···d88P················888b·····d888·d8b·······················
%  ·····Y88b·d88P·················8888b···d8888·Y8P·······················
%  ······Y88o88P··················88888b·d88888···························
%  ·······Y888P··8888b···88888b···888Y88888P888·888·88888b·····d88b·······
%  ········888······"88b·888·"88b·888·Y888P·888·888·888·"88b·d88P"88b·····
%  ········888···d888888·888··888·888··Y8P··888·888·888··888·888··888·····
%  ········888··888··888·888··888·888···"···888·888·888··888·Y88b·888·····
%  ········888··"Y888888·888··888·888·······888·888·888··888··"Y88888·····
%  ·······························································888·····
%  ··························································Y8b·d88P·····
%  ···························································"Y88P"······
%  =======================================================================
% 
%  -----------------------------------------------------------------------
% Author       : 焱铭
% Date         : 2023-07-18 08:50:38 +0800
% LastEditTime : 2025-04-20 23:08:23 +0800
% Github       : https://github.com/YanMing-lxb/
% FilePath     : /GUET_Thesis_LaTeX/main.tex
% Description  : Version 2.11.9 更新请关注 https://github.com/YanMing-lxb/GUET_Thesis_LaTeX
%  -----------------------------------------------------------------------
%

% \special{dvipdfmx:config z 0}                                                % XeLaTeX取消PDF压缩,加快编译速度,但会增加PDF体积
% \pdfcompresslevel=0                                                          % PdfLaTeX取消PDF压缩,加快编译速度,但会增加PDF体积
% \pdfobjcompresslevel=0                                                       % LuaLaTeX取消PDF压缩,加快编译速度,但会增加PDF体积

% ------------------------------------------------------------------------------------------
%     前言区域
% ------------------------------------------------------------------------------------------

\documentclass[master]{GUET-Thesis} 
% pversion:打印版 bversion:盲审版 
% bachelor:本科 master:学硕 promaster:专硕 doctor:博士 ojmaster 在职硕士 ptmaster 非全专硕

% \documentclass[master,bversion]{GUET-Thesis} % 盲审版
% \documentclass[master,pversion]{GUET-Thesis} % 打印版

% ------------------------------------------------------------------------------------------
%     资源路径
% ------------------------------------------------------------------------------------------

\graphicspath{
        {./Pictures/},
        {./Pictures/Chapter1/},
        {./Pictures/Chapter2/},
        {./Pictures/Chapter3/},
        {./Pictures/Chapter4/},
        {./Pictures/Chapter5/}
}                                                                           % 图片所在位置,根据需求进行修改
\ThesisBibResource{./References/References.bib}                              % 参考文献数据源加载
\ThesisAchResource{./References/Accomplishs.bib}                             % 攻读学位期间取得成果数据源加载

% ------------------------------------------------------------------------------------------
%     封面信息
% ------------------------------------------------------------------------------------------

\Title{                                                                     % 题目{中文}{英文}  
    基于嵌入式散热模块的微通道散热技术研究
}{
    Research on microchannel heat dissipation technology \\& based on embedded heat dissipation module
    }                                                                       % 标题中插入“\\&”命令进行换行
\Author{焱铭}                                                               % 作者姓名
\Advisor{XXX}                                                             % 导师姓名
\Protitle{教授}                                                              % 导师职称
\School{机电工程学院}                                                        % 所在学院
\Major{机械工程}                                                             % 学科专业或领域
\ResearchDirection{电子器件散热}                                             % 研究领域(盲审用)
\DegreeCategories{工学硕士}                                                  % 申请学位门类或类别
\StudentNumber{2020XXXXX}                                                     % 学号
\Secrets{}                                                                  % 密级,不涉密请空着
\Date{\today}                                                               % 可更换为具体日期如:\date{2023年5月28日}
% \subCrefNameSet                                                             % 子图、表、算法等引用名称格式设置为 子图 2-1(a) 的样式
\SymbolGroupStyle{符号分类}                                                  % 符号分组样式,可选参数为“符号分类”或“章节分类”
\theAlgoSet{algorithmicx}                                                   % 设置算法环境,默认为 algorithmicx,可选参数为 algorithm2e 或 algorithmicx ,注意启用不同宏包时需要适用对应算法命令
% ------------------------------------------------------------------------------------------
%     正文
% ------------------------------------------------------------------------------------------

\begin{document}
    \frontmatter % 前言部分
    \MakeCover                                                             % 封面
    \OriginalityDeclaration                                                % 独创性声明
    % \SignatureDeclaration{./Chapters/独创性声明(示例).pdf}                % 可使用已签字的独创性声明PDF文件
    
    % !Mode:: "TeX:UTF-8"


\begin{chineseabstract}
    LaTeX利用设置好的模板,可以编译为格式统一的pdf。目前国内大多出版社与高校仍在使用word,word由于其强大的功能与灵活性,在新手面对形式固定的论文时,排版、编号、参考文献等简单事务反而会带来很多困难与麻烦,对于一些需要通篇修改的问题,要想达到LaTeX的效率,对word使用者来说需要具有较高的技能水平。

    为了能把主要精力放在论文撰写上,许多国际期刊和高校都支持LaTeX的撰写与提交,新手不需要关心格式问题,只需要按部就班的使用少数符号标签,即可得到符合要求的文档。且在需要全篇格式修改时,更换或修改模板文件,即可直接重新编译为新的样式文档,这对于word新手使用word的感受来说是不可思议的。
    此项目提供用于排版桂林电子科技大学本硕博(非全,在职)毕业论文(设计)LaTeX模板类,旨在帮助桂林电子科技大学的毕业生高效地完成毕业论文的写作。

\chinesekeyword{桂林电子科技大学;本硕博学位论文;\LaTeX 模板类;\LaTeXe}
\end{chineseabstract}


\begin{englishabstract}
    LaTeX can be compiled into a pdf of uniform format using the set template.At present, most domestic publishers and universities still use word. Because of its powerful function and flexibility, when faced with fixed-form papers by novices, simple matters such as typesetting, numbering, and reference documents will bring many difficulties and troubles. For some problems that need to be modified throughout, to achieve the efficiency of LaTeX, it requires a high level of skill for word users.

    In order to focus on the writing of papers, many international journals and universities support the writing and submission of LaTeX. Novices don't need to care about formatting issues. They only need to use a few symbolic labels step by step to get the documents that meet the requirements. And when you need to modify the entire format, you can directly recompile the template file by replacing or modifying the template file. This is incredible for the word novice to use the word.


\englishkeyword{GUET; Common template; \LaTeX; \LaTeXe}
\end{englishabstract}                                              % 摘要

% ------------------------------------------------------------------------------------------ 

    \ThesisFigureList                                                      % 插图目录
    \ThesisTableList                                                       % 插表目录
    \ThesisSymbolList                                                      % 符号说明表
    \ThesisContents                                                        % 目录

% ------------------------------------------------------------------------------------------
    \mainmatter % 正文部分

    % 对于需要引言并且不需要章节编号且要放在正文中的情况可以使用以下命令
    % \thispagestyle{empty}                                                  % 保留原有空白页设置
    % \StandardHeader                                                        % 手动应用标准页眉
    % \chapter*{\texorpdfstring{引\quad 言}{引言}}
    % \addcontentsline{toc}{chapter}{\texorpdfstring{引\quad 言}{引言}}        % 添加到目录中
    
    % A组表示希腊字符说明,B组为下标说明,C组为缩略词说明。
\nomenclature[C,01]{PCB}{Printed Circuit Board}
\nomenclature[C,02]{LTCC}{Low temperature cofired ceramic}
\nomenclature[C,03]{MATD}{mean absolute temperature deviation}

\nomenclature[]{$T_{f}$}{流体温度 $(K)$}
\nomenclature[]{$T_{s}$}{固体温度 $(K)$}

\nomenclature[A]{$\rho_{f}$}{流体密度 $(kg/m^3)$}
\nomenclature[A]{$\mu_{f}$}{流体动力粘度 $(kg/(m \cdot s))$}

\nomenclature[B]{$s$}{固体}
\nomenclature[B]{$f$}{流体}
% \nomenclature[B]{$$}{}                                                       % 符号定义文件

    % !Mode:: "TeX:UTF-8"
%此为第一章节。
%[h]为hear代码所在位置,\caption为表注题注,\cref{}引用图表公式章节等,\cite为引用参考文献,\subfloat子图,\label标签,\begin{figure}图片环境,\begin{table}表格环境,\begin{equation}公式环境,\toprule三线表顶线,\cmidrule三线表中线,\bottomrule三线表底线,\begin{theorem}定理,\begin{proof}证明,\begin{corollary}推论,\begin{lemma}引理
    
\chapter{关于模板的说明}\label{ch:1}


\section{如何看本文档}
本文档简单介绍了模板的一些基础使用方法,在阅读本文档时,应当将代码与PDF文档对照来看,了解各个代码所对应的实现效果。

\begin{shaded}
    该命令可用于提醒自己,该段的内容中心。(可以删掉!)
\end{shaded}

\section{环境配置与模板参数说明}
请详细阅读本项目根目录下的README.md 文档

\section{节标题示例}

\subsection{小节标题}

\subsubsection{小小节标题}

\section{参考文献插入示例}

    \LaTeX 插入参考文献只需在\textbackslash{cite\{\}}中输入文献的key即可,如:示例\cite{Lau_2022}。如需插入多个参考文献只需用英文逗号分隔即可,\LaTeX 会自动进行排序整理文献,如:示例\cite{Sadique.Murtaza.ea_2022, Tan.Du.ea_2021, Lau_2022}

    实际应用如下:

    随着人工智能和第五代移动通信技术等系统技术的发展\cite{Lau_2022},推动着半导体行业在移动便携设备、高性能计算机、自动驾驶、物联网和大数据等应用领域的发展\cite{Lau_2022},同时也推动着电子芯片向着小型化和高集成化方向发展快速发展\cite{Lau_2022,Sadique.Murtaza.ea_2022, Tan.Du.ea_2021}。
……

\section{列表示例}

\subsection{纯数字编号}
\begin{enumerate}
 \item XXXXXXXXXX
 \label{item1}
 \item XXXXXXXXXX
 \item XXXXXXXXXX
\end{enumerate}

\subsection{罗马编号}
\begin{enumerate}[label=(\roman*)]
 \item XXXXXXXXXX
 \label{item2}
 \item XXXXXXXXXX
 \item XXXXXXXXXX
\end{enumerate}

\subsection{括号编号}
\begin{enumerate}[label=(\arabic*)]
 \item XXXXXXXXXX
 \label{item3}
 \item XXXXXXXXXX
 \item XXXXXXXXXX
\end{enumerate}

\subsection{半括号编号}
\begin{enumerate}[label=\arabic*)]
 \item XXXXXXXXXX
 \label{item4}
 \item XXXXXXXXXX
 \item XXXXXXXXXX
\end{enumerate}

\subsection{小字母编号}
\begin{enumerate}[label=\alph*)]
 \item XXXXXXXXXX
 \label{item5}
 \item XXXXXXXXXX
 \item XXXXXXXXXX
\end{enumerate}

\subsection{自定义编号}
\begin{enumerate}[leftmargin = 6em, labelsep = 0em]
    \item[步骤一、] XXXXXXXXXX;
    \item[步骤二、] XXXXXXXXXX;
    \item[步骤三、] XXXXXXXXXX;
\end{enumerate}

引用测试,正如\cref{item1}、\cref{item2}、\cref{item3}、\cref{item4}、\cref{item5}所示


\section{本论文的结构安排}
\cref{ch:1}:绪论。本章主要进行整体说明。

\cref{ch:2}:图片示例。

\cref{ch:3}:表格示例。

\cref{ch:4}:数学公式示例。

\cref{ch:5}:列表、算法、定理、证明插入示例。

\cref{ch:6}:全文总结与展望。本次研究工作进行总结,并根据全文研究过程中……。



    % !Mode:: "TeX:UTF-8"
%此为章节二模板
%\chapter、\section、\subsection、\subsubsection分别对应一二三四级标题
\chapter{图片示例}\label{ch:2}

\section{图片排版示例}
\textbf{注意}:使用\textbackslash caption[]\{\}命令时,如果不需要设置缩写目录的内容,一定要删掉[],否则插图插表索引将不会显示该图或表的目录。

\textbf{建议}:在论文写作时图片位置可以先按照写作时的习惯进行放置,待到完成所有写作内容后再进行详细调整图片位置。

\subsection{图片格式}

\LaTeX 中图片推荐使用pdf格式。使用Origin可导出矢量无白边的图片以保证清晰度,其次推荐使用jpg,png格式图片。

为了保证图片的清晰,jpg图片导出时ppi建议设置为300,png图片导出时建议宽度设置为1024像素(可根据需求自行设置),长度随宽度变化。

\subsection{单图排版示例}

\begin{figure}[htb]
    \includegraphics[width=0.8 \textwidth]{50-years-processor-trend.png}
    \caption[处理器发展]{近50年微处理器发展趋势} % 中括号中内容为插图索引中显示内容,可在题注内容过长时使用
    \label{fig:processor-trend}
\end{figure}

图片引用示例:\cref{fig:processor-trend}。

\subsection{多图排版示例}
同一行中的子图之间要留有空间,不要占满!否则会自动换行!

子图之间空一行表示换行。

插入子图请使用\textbackslash subfloat\{\}命令。

\begin{figure}[htb]
    \subfloat[改进前的结构]{
        \label{fig:Unimproved-cooling-structure}
        \includegraphics[width=0.45\linewidth]{Unimproved-cooling-structure.png}}
    \subfloat[基板内进行微通道散热]{
        \label{fig:LTCC-Microchannels}
        \includegraphics[width=0.45\linewidth]{LTCC-Microchannels.png}}

    \subfloat[嵌入散热模块]{
        \label{fig:Embedded-cooling-module}
        \includegraphics[width=0.45\linewidth]{Embedded-cooling-module.png}}
    \subfloat[带针鳍或肋的嵌入式散热模块]{
        \label{fig:Rib-pin-fin}
        \includegraphics[width=0.45\linewidth]{Rib-pin-fin.png}}
    \caption{三种强化传热途径示意图}
    \label{fig:Three-enhanced-heat-transfer-paths}
\end{figure}

子图引用示例:\cref{fig:Unimproved-cooling-structure},

整图引用示例:\cref{fig:Three-enhanced-heat-transfer-paths}。

提供三种不同的Tikz绘图示例,分别为柱状图(存在问题是少一列,过几天修改),多点折线图,少点折线图(数据均已进行扰动,有空还是要大改一下数据)。

\begin{figure}
    \subfloat[生成证明\label{fig-2}]{
        \begin{tikzpicture}[global scale = 0.55]
            \begin{axis}[
                x tick label style={
                    /pgf/number format/1000 sep=.},
                y tick label style={
                    /pgf/number format/1000 sep=.},
                xlabel={参与者数量},
                ylabel={耗时 (ms)},
                ymin=0, ymax=1100,
                ybar=10pt,
                bar width=20pt,
                enlarge x limits={abs=6pt},
                legend cell align={left},
                enlarge y limits={value=0.1,upper},
                ybar interval=0.7,
                legend pos=north west,
                xtick=data,
                xticklabels={100, 200, 300, 400, 500},
                ytick={0, 500, 1000},
                legend style={font=\huge},
                label style={font=\huge},
                tick label style={font=\huge}
                ]
                \addplot[color=orange,fill=orange!50,bar shift=-6pt] coordinates {
                    (1,166.7)
                    (2,334.6)
                    (3,502.1)
                    (4,667.2)
                    (5,831.4)
                };
                \addplot[color=blue,fill=blue!50,bar shift=-2pt] coordinates {
                    (1,234.3)
                    (2,501.5)
                    (3,768.7)
                    (4,1035.9)
                    (5,1303.1)
                };
                \addplot[color=green,fill=green!50,bar shift=0pt] coordinates {
                    (1,265.824)
                    (2,533.074)
                    (3,800.234)
                    (4,1067.464)
                    (5,1334.614)
                };
                \addplot[color=gray,fill=gray!50,bar shift=2pt] coordinates {
                    (1,130.214)
                    (2,261.854)
                    (3,393.414)
                    (4,525.034)
                    (5,656.674)
                };
                \addplot[color=red,fill=red!50,bar shift=6pt] coordinates {
                    (1,117.529)
                    (2,236.439)
                    (3,343.239)
                    (4,470.919)
                    (5,588.179)
                };
                \legend{Xu et al.\cite{xu2019verifynet},Zhang et al.\cite{zhang2020privacy},Shen et al.\cite{shen2018enabling},Han et al.\cite{han2022verifiable},本方案}
            \end{axis}
        \end{tikzpicture}
    }
    \hfill
    \subfloat[验证证明\label{fig-3}]{
        \begin{tikzpicture}[global scale = 0.55]
            \begin{axis}[
                x tick label style={
                    /pgf/number format/1000 sep=.},
                y tick label style={
                    /pgf/number format/1000 sep=.},
                xlabel={参与者数量},
                ylabel={耗时 (ms)},
                ymin=0, ymax=2700,
                ybar=10pt,
                bar width=20pt,
                enlarge x limits={abs=6pt},
                legend cell align={left},
                enlarge y limits={value=0.1,upper},
                ybar interval=0.7,
                legend pos=north west,
                xtick=data,
                xticklabels={100, 200, 300, 400, 500},
                legend style={font=\huge},
                label style={font=\huge},
                tick label style={font=\huge}
                ]
                \addplot[color=orange,fill=orange!50,bar shift=-6pt] coordinates {
                    (1,216.316)
                    (2,291.916)
                    (3,362.116)
                    (4,431.616)
                    (5,502.716)
                };
                \addplot[color=blue,fill=blue!50,bar shift=-10pt] coordinates {
                    (1,1169.397)
                    (2,1436.597)
                    (3,1703.797)
                    (4,1970.997)
                    (5,2238.197)
                };
                \addplot[color=green,fill=green!50,bar shift=-5pt] coordinates {
                    (1,361.874)
                    (2,629.074)
                    (3,896.274)
                    (4,1163.474)
                    (5,1430.674)
                };
                \addplot[color=gray,fill=gray!50,bar shift=5pt] coordinates {
                    (1,779.338)
                    (2,1445.338)
                    (3,2111.338)
                    (4,2777.338)
                    (5,3443.338)
                };
                \addplot[color=red,fill=red!50,bar shift=10pt] coordinates {
                    (1,80.632)
                    (2,160.932)
                    (3,241.232)
                    (4,321.532)
                    (5,401.832)
                };
                \legend{XXX et al.\cite{Lau_2022},YYY et al.\cite{Lau_2022},ZZZ et al.\cite{Lau_2022},HHH et al.\cite{Lau_2022},本方案}
            \end{axis}
        \end{tikzpicture}
    }
    \caption[方案一的开销比较]{比较结果 (a) 生成证明的计算开销比较 (b)验证的计算开销比较(在\cref{fig-3}中,Zhang 等人的方案中的 $d=25$~\cite{zhang2020privacy} 和 Shen 等人的方案中的 $l=10$\cite{shen2018enabling})}
    \label{fig3}
\end{figure}

在\cref{fig-2}和\cref{fig-3}中,是引用。

\begin{figure}[tbh!]
    \subfloat[MNIST\label{fig-4}]{
        \begin{tikzpicture}[global scale = 0.6]
            \begin{axis}[
                xlabel={迭代次数},
                ylabel={准确率 (\%)},
                xticklabel style={anchor= east,rotate=45 },
                xmax=250,
                xmin=0,
                ymax=100,
                ymin=38,
                legend pos=south east,
                ymajorgrids=true,
                grid style=dashed,
                legend style={nodes={scale=1, transform shape}},
                legend style={font=\huge},
                label style={font=\huge},
                tick label style={font=\huge}
                ]
                \addplot[
                color=blue,
                mark=.,
                only marks,
                sharp plot,
                line width=1.5pt
                ] table [x index=0, y index=1] {Data/data_sub_2.dat};
                \addplot[
                color=green,
                mark=.,
                only marks,
                sharp plot,
                line width=1.5pt
                ] table [x index=0, y index=2] {Data/data_sub_2.dat};
                \addplot[
                color=red,
                mark=.,
                only marks,
                sharp plot,
                line width=1.5pt
                ] table [x index=0, y index=3] {Data/data_sub_2.dat};
                \legend{100 参与者, 300 参与者, 500 参与者}
            \end{axis}
        \end{tikzpicture}
    }
    \hfill
    \subfloat[CIFSAR{\footnotesize 100}\label{fig-5}]{
        \begin{tikzpicture}[global scale = 0.6]
            \begin{axis}[
                xlabel={迭代次数},
                ylabel={准确率 (\%)},
                xticklabel style={anchor= east,rotate=45 },
                xmax=250,
                xmin=0,
                ymax=80,
                ymin=23,
                legend pos=south east,
                ymajorgrids=true,
                grid style=dashed,
                legend style={nodes={scale=1, transform shape}},
                legend style={font=\huge},
                label style={font=\huge},
                tick label style={font=\huge}
                ]
                \addplot[
                color=blue,
                mark=.,
                only marks,
                sharp plot,
                line width=1.5pt
                ] table [x index=0, y index=1] {Data/data_sub_1.dat};
                \addplot[
                color=green,
                mark=.,
                only marks,
                sharp plot,
                line width=1.5pt
                ] table [x index=0, y index=2] {Data/data_sub_1.dat};
                \addplot[
                color=red,
                mark=.,
                only marks,
                sharp plot,
                line width=1.5pt
                ] table [x index=0, y index=3] {Data/data_sub_1.dat};
                \legend{100 参与者, 300 参与者, 500 参与者}
            \end{axis}
        \end{tikzpicture}
    }
    \caption[方案一的准确度与迭代次数关系]{不同数量的客户端的准确性与迭代次数的关系}
    \label{exp1}
\end{figure}

\begin{figure}[tbh!]
    \subfloat[MNIST\label{fig-6}]{
        \begin{tikzpicture}[global scale = 0.6]
            \begin{axis}[
                xlabel={参与者数量},
                ylabel={准确率 (\%)},
                symbolic x coords = {100,200,300,400,500},
                xticklabel style={anchor= east,rotate=45 },
                xtick=data,
                ymax=100,
                ymin=40,
                legend pos=south east,
                legend cell align={left},
                ymajorgrids=true,
                grid style=dashed,
                legend style={nodes={scale=1, transform shape}},
                legend style={font=\huge},
                label style={font=\huge},
                tick label style={font=\huge}
                ]
                \addplot+[mark size=3pt]
                coordinates {
                    (100, 91.0349)
                    (200, 91.7635)
                    (300, 92.1530)
                    (400, 93.5713)
                    (500, 94.2747)
                };\label{6c0}%原始
                \addplot+[mark size=3pt]
                coordinates {
                    (100, 88.4214)
                    (200, 88.6571)
                    (300, 89.5672)
                    (400, 90.6631)
                    (500, 91.2861)
                };\label{6c1}%Xu
                \addplot+[mark size=3pt]
                coordinates {
                    (100, 86.8072)
                    (200, 87.0289)
                    (300, 88.9112)
                    (400, 89.7785)
                    (500, 90.0550)
                };\label{6c2}%Zhang
                \addplot+[mark size=3pt]
                coordinates {
                    (100, 90.5275)
                    (200, 90.1048)
                    (300, 91.8314)
                    (400, 92.4257)
                    (500, 93.5617)
                };\label{6c3}%Shen
                \addplot+[mark size=3pt]
                coordinates {
                    (100, 91.2241)
                    (200, 91.7189)
                    (300, 92.6760)
                    (400, 93.1523)
                    (500, 93.8426)
                };\label{6c4}%Han
                \addplot+[mark size=3pt]
                coordinates {
                    (100, 86.4234)
                    (200, 88.3281)
                    (300, 89.5347)
                    (400, 90.2071)
                    (500, 91.8193)
                };\label{6c5}%5%
                \addplot+[mark size=3pt]
                coordinates {
                    (100, 82.6479)
                    (200, 85.4558)
                    (300, 87.1009)
                    (400, 88.3564)
                    (500, 89.6152)
                };\label{6c6}%10%
                \addplot+[mark size=3pt]
                coordinates {
                    (100, 73.4036)
                    (200, 76.6815)
                    (300, 77.5426)
                    (400, 78.1326)
                    (500, 78.9213)
                };\label{6c7}%20%
            \legend{本方案, XXX et al.\cite{Lau_2022}, YYY et al.\cite{Lau_2022}, ZZZ et al.\cite{Lau_2022}, HHH et al.\cite{Lau_2022}, 5\% 恶意参与者, 10\% 恶意参与者, 20\% 恶意参与者}
            \end{axis}
        \end{tikzpicture}
    }
    \hfill
    \subfloat[CIFSAR{\footnotesize 100}\label{fig-7}]{
        \begin{tikzpicture}[global scale = 0.6]
            \begin{axis}[
                %title={},
                xlabel={参与者数量},
                ylabel={准确率 (\%)},
                symbolic x coords = {100,200,300,400,500},
                xticklabel style={anchor= east,rotate=45 },
                xtick=data,
                ymax=80,
                ymin=20,
                legend pos=south east,
                legend cell align={left},
                ymajorgrids=true,
                grid style=dashed,
                legend style={nodes={scale=1, transform shape}},
                legend style={font=\huge},
                label style={font=\huge},
                tick label style={font=\huge}
                ]
                \addplot+[mark size=3pt]
                coordinates {
                    (100, 73.0039)
                    (200, 73.4271)
                    (300, 73.7160)
                    (400, 74.2513)
                    (500, 75.0248)
                };%原始
                \addplot+[mark size=3pt]
                coordinates {
                    (100, 70.5137)
                    (200, 71.2461)
                    (300, 71.5914)
                    (400, 72.6225)
                    (500, 73.3046)
                };%Xu
                \addplot+[mark size=3pt]
                coordinates {
                    (100, 70.0164)
                    (200, 71.2286)
                    (300, 72.1332)
                    (400, 72.4517)
                    (500, 73.1561)
                };%Zhang
                \addplot+[mark size=3pt]
                coordinates {
                    (100, 72.3538)
                    (200, 72.6216)
                    (300, 73.8918)
                    (400, 74.2905)
                    (500, 74.1621)
                };%Shen
                \addplot+[mark size=3pt]
                coordinates {
                    (100, 72.1013)
                    (200, 73.5135)
                    (300, 73.9425)
                    (400, 74.7297)
                    (500, 75.2756)
                };%Han
                \addplot+[mark size=3pt]
                coordinates {
                    (100, 68.5217)
                    (200, 70.4832)
                    (300, 71.7219)
                    (400, 72.0186)
                    (500, 72.3318)
                };%5%
                \addplot+[mark size=3pt]
                coordinates {
                    (100, 62.2815)
                    (200, 65.8233)
                    (300, 67.6183)
                    (400, 68.1124)
                    (500, 69.5276)
                };%10%
                \addplot+[mark size=3pt]
                coordinates {
                    (100, 53.6133)
                    (200, 55.5052)
                    (300, 58.4218)
                    (400, 61.1081)
                    (500, 62.3716)
                };%20%
                \legend{本方案, XXX et al.\cite{Lau_2022}, YYY et al.\cite{Lau_2022}, ZZZ et al.\cite{Lau_2022}, HHH et al.\cite{Lau_2022}, 5\% 恶意参与者, 10\% 恶意参与者, 20\% 恶意参与者}
            \end{axis}
        \end{tikzpicture}
    }
    \caption[方案一的准确度比较]{准确度比较}
    \label{exp2}
\end{figure}

引用也是一样的,如\cref{fig-4}和\cref{fig-5}。和\cref{fig-6}和\cref{fig-7}。以及对整个图的引用\cref{exp1}和\cref{exp2}。

\section{本章小节}
本章介绍了基于嵌入式散热模块的微通道散热技术所涉及的基……
    % !Mode:: "TeX:UTF-8"

\chapter{表格示例}\label{ch:3}
可使用excel绘制表格,然后粘贴到以下网站中生成 \LaTeX 表格代码,然后再在网站中进行详细调整

推荐网站如下:

https://www.tablesgenerator.com/

https://www.latex-tables.com/


\section{普通三线表示例}
普遍学者认为,微通道指的是水力直径在 $10\ \mathrm{\mu m}$ 到 $1000\ \mathrm{\mu m}$ 范围内的通道(也有观点认为是 $1\ \mathrm{\mu m}$ 到 $100\ \mathrm{\mu m}$)所构成的换热器。
以下是较为常见的微通道尺寸分类,可以参见\cref{tab:division-of-microchannels}。
\begin{table}[htbp]
    \caption[微通道的划分]{微通道的划分\cite{LuSiHong_2021}}
    \setlength{\tabcolsep}{14mm}{ % 因表格过窄,手动设置宽度为7mm
        \begin{tabular}{lc}
            \toprule
            通道种类    & 水力直径$\mu m$   \\
            \midrule
            分子纳米通道  & $\le 0.1$     \\
            过渡性纳米通道 & $0.1\sim 1$   \\
            过渡性微通道  & $1\sim 10$    \\
            微通道     & $10\sim 1000$ \\
            常规通道    & $>1000$       \\
            \bottomrule
        \end{tabular}}
    \label{tab:division-of-microchannels}
\end{table}

\begin{table}[htbp]
    \centering
    \caption{三线表示例(tabularray自定义环境)}
    \begin{threetab}{
        colspec = {cc}, 
        column{1} = {4cm}, % 设置第一列宽度
        column{2} = {5cm}, % 设置第二列宽度
        }
        表头1  & 表头2 \\
        内容1  & 内容2 \\
        内容3  & 内容4 \\
    \end{threetab}
\end{table}

\begin{table}[!htbp]
    \caption[表格复杂定义示例]{复杂定义示例(提供脚注)}
    \begin{threeparttable}
        \begin{tabular}{*{3}{p{3.3cm}<{\centering}}}
            \hline
            方案 & 参数1 & 参数2\\ \hline
            xxx et al.~\cite{Lau_2022} & $2n\times$SS & — \\ 
            YYY~\cite{Lau_2022} & $n\times$SS$+5n\times$P & $n\times$BP \\
            ZZZ~\cite{Lau_2022} & $2n\times$SS$+n\times$P & $n\times$BP$+n\times$SS \\
            本章方案\textsuperscript{1} & $n\times$SS & $n\times$P \\ \hline
        \end{tabular}
        \begin{tablenotes}
            \footnotesize
            \item[1] 服务器发起的最后一个操作被视为聚合部分
        \end{tablenotes}
    \end{threeparttable}
\end{table}


\section{子表排版示例}
\begin{table}[htb]
    \centering
    \begin{subtable}{0.45\textwidth}
        \centering
        \begin{threetab}{
            colspec = {cc}, 
            }
            表头1  & 表头2 \\
            内容1  & 内容2 \\
            内容3  & 内容4 \\
        \end{threetab}
        \caption{子表1标题}
    \end{subtable}
    \quad
    \begin{subtable}{0.45\textwidth}
        \centering
        \begin{threetab}{
            colspec = {cc}, 
            }
            表头1  & 表头2 \\
            内容1  & 内容2 \\
            内容3  & 内容4 \\
        \end{threetab}
        \caption{子表2标题}
    \end{subtable}
    \caption{主表标题}
\end{table}

\section{跨页表格示例}

\begin{longtable}{@{\extracolsep{\fill}}cccccc@{}}  \\ % @{\extracolsep{\fill}}cccccc@{}命令表格整体宽度为页面宽
    \caption{RSM仿真实验规划表}
    \label{tab:Experimental-Planning}  \\
    \toprule
    标准序 & 运行序 & $H_{rib}\ \mathrm{(mm)}$ & $H_{pf}\ \mathrm{(mm)}$ & $N_{pf}$ & $N_{ac}$ \\ \midrule
    \endfirsthead
    %
    \multicolumn{6}{c}%
    {{表 \thetable\ RSM仿真实验规划表 (续)}} \\
    \toprule
    标准序 & 运行序 & $H_{rib}\ \mathrm{(mm)}$ & $H_{pf}\ \mathrm{(mm)}$ & $N_{pf}$ & $N_{ac}$ \\ \midrule
    \endhead
    %
    \bottomrule
    \endfoot
    %
    \endlastfoot
    %
    11  & 1   & 0.16            & 0.8            & 6        & 16       \\
    13  & 2   & 0.16            & 0.16           & 22       & 16       \\
    15  & 3   & 0.16            & 0.8            & 22       & 16       \\
    12  & 4   & 0.8             & 0.8            & 6        & 16       \\
    10  & 5   & 0.8             & 0.16           & 6        & 16       \\
    2   & 6   & 0.8             & 0.16           & 6        & 0        \\
    19  & 7   & 0.48            & 0.48           & 14       & 8        \\
    1   & 8   & 0.16            & 0.16           & 6        & 0        \\
    20  & 9   & 0.48            & 0.48           & 14       & 8        \\
    18  & 10  & 0.48            & 0.48           & 14       & 8        \\
    8   & 11  & 0.8             & 0.8            & 22       & 0        \\
    14  & 12  & 0.8             & 0.16           & 22       & 16       \\
    6   & 13  & 0.8             & 0.16           & 22       & 0        \\
    17  & 14  & 0.48            & 0.48           & 14       & 8        \\
    7   & 15  & 0.16            & 0.8            & 22       & 0        \\
    16  & 16  & 0.8             & 0.8            & 22       & 16       \\
    4   & 17  & 0.8             & 0.8            & 6        & 0        \\
    9   & 18  & 0.16            & 0.16           & 6        & 16       \\
    5   & 19  & 0.16            & 0.16           & 22       & 0        \\
    3   & 20  & 0.16            & 0.8            & 6        & 0        \\
    25  & 21  & 0.48            & 0.48           & 6        & 8        \\
    22  & 22  & 0.8             & 0.48           & 14       & 8        \\
    23  & 23  & 0.48            & 0.16           & 14       & 8        \\
    29  & 24  & 0.48            & 0.48           & 14       & 8        \\
    28  & 25  & 0.48            & 0.48           & 14       & 16       \\
    30  & 26  & 0.48            & 0.48           & 14       & 8        \\
    26  & 27  & 0.48            & 0.48           & 22       & 8        \\
    27  & 28  & 0.48            & 0.48           & 14       & 0        \\
    21  & 29  & 0.16            & 0.48           & 14       & 8        \\
    24  & 30  & 0.48            & 0.8            & 14       & 8        \\ \bottomrule
\end{longtable}

    %
%  =======================================================================
%  ····Y88b···d88P················888b·····d888·d8b·······················
%  ·····Y88b·d88P·················8888b···d8888·Y8P·······················
%  ······Y88o88P··················88888b·d88888···························
%  ·······Y888P··8888b···88888b···888Y88888P888·888·88888b·····d88b·······
%  ········888······"88b·888·"88b·888·Y888P·888·888·888·"88b·d88P"88b·····
%  ········888···d888888·888··888·888··Y8P··888·888·888··888·888··888·····
%  ········888··888··888·888··888·888···"···888·888·888··888·Y88b·888·····
%  ········888··"Y888888·888··888·888·······888·888·888··888··"Y88888·····
%  ·······························································888·····
%  ··························································Y8b·d88P·····
%  ···························································"Y88P"······
%  =======================================================================
% 
%  -----------------------------------------------------------------------
% Author       : 焱铭
% Date         : 2024-05-31 22:39:58 +0800
% LastEditTime : 2025-02-07 14:27:36 +0800
% Github       : https://github.com/YanMing-lxb/
% FilePath     : /GUET_Thesis_LaTeX/Chapters/Chapter4.tex
% Description  : 
%  -----------------------------------------------------------------------
%

% !Mode:: "TeX:UTF-8"

\chapter{数学公式示例}\label{ch:4}

本章在\cref{ch:3}完成基……。

\section{公式示例}
在本次研究中应用到计算流体动力学(Computational Fluid Dynamics,CFD)对研究对象进……。


\subsection{普通带序号公式}
\begin{equation}
    \frac{\partial u}{\partial x}+\frac{\partial v}{\partial y}+\frac{\partial v}{\partial z}=0
\end{equation}
u,v,w 分别是 x,y,z 方向的速度分量。


\subsection{需要对齐的多个带序号的公式}

\& 号为对其对齐标记最好放置在计算符号之前,如=、+、-之前;

\verb|\\| 表示换行。

\begin{align}% 式中的&为对齐的位置标记
    u & \frac{\partial u}{\partial x}+v \frac{\partial u}{\partial y}+w \frac{\partial u}{\partial z}=-\frac{1}{\rho_{f}} \frac{\partial p}{\partial x}+\frac{\mu_{f}}{\rho_{f}}\left(\frac{\partial^{2} u}{\partial x^{2}}+\frac{\partial^{2} u}{\partial y^{2}}+\frac{\partial^{2} u}{\partial z^{2}}\right) \\
    u & \frac{\partial v}{\partial x}+v \frac{\partial v}{\partial y}+w \frac{\partial v}{\partial z}=-\frac{1}{\rho_{f}} \frac{\partial p}{\partial y}+\frac{\mu_{f}}{\rho_{f}}\left(\frac{\partial^{2} v}{\partial x^{2}}+\frac{\partial^{2} v}{\partial y^{3}}+\frac{\partial^{2} v}{\partial z^{3}}\right) \\
    u & \frac{\partial w}{\partial x}+v \frac{\partial w}{\partial y}+w \frac{\partial w}{\partial z}=-\frac{1}{\rho_{f}} \frac{\partial p}{\partial z}+\frac{\mu_{f}}{\rho_{f}}\left(\frac{\partial^{2} w}{\partial x^{2}}+\frac{\partial^{2} w}{\partial y^{2}}+\frac{\partial^{2} w}{\partial z^{2}}\right)
\end{align}
$\rho_{f}$ 和 $\mu_{f}$ 分别是冷却剂的密度和动态粘度,p 是冷却剂压力。


\subsection{需要换行对齐的长公式}

\begin{equation}\label{eq:P}
    \begin{split}
        f^{\prime} & = 6.272 + 3.02 A + 6.08 B + 0.0368 C - 0.8848 D  \\
            & + 0.04381 D^2 + 6.35 AB - 0.3602 AD - 0.5497 BD
    \end{split}
\end{equation}

\subsection{其他公式示例}
\begin{equation}
    \begin{aligned}
    \left\{
        \begin{array}{l}
        \text {find}\enspace H_{rib},H_{rib},N_{pf},N_{ac} \\
        \text {min} \enspace F(H_{rib},H_{rib},N_{pf},N_{ac})= min\{f_1,f_2,f_3\} \\

            \text{s.t.\enspace}\enspace 0.2 \leqslant H_{rib} \leqslant 0.8    \\
            \hspace{2.2em} 0.2 \leqslant H_{pf} \leqslant 0.8                     \\
            \hspace{2.2em} 6 \leqslant N_{pf} \leqslant 22,\ N_{pf}\in \mathbb{O} \\
            \hspace{2.2em} 0 \leqslant N_{ac} \leqslant 16,\ N_{ac}\in \mathbb{E}
        \end{array}
    \right. 
    \end{aligned}
    \label{eq:MO}
\end{equation}

\section{本章小节}
    %
%  =======================================================================
%  ····Y88b···d88P················888b·····d888·d8b·······················
%  ·····Y88b·d88P·················8888b···d8888·Y8P·······················
%  ······Y88o88P··················88888b·d88888···························
%  ·······Y888P··8888b···88888b···888Y88888P888·888·88888b·····d88b·······
%  ········888······"88b·888·"88b·888·Y888P·888·888·888·"88b·d88P"88b·····
%  ········888···d888888·888··888·888··Y8P··888·888·888··888·888··888·····
%  ········888··888··888·888··888·888···"···888·888·888··888·Y88b·888·····
%  ········888··"Y888888·888··888·888·······888·888·888··888··"Y88888·····
%  ·······························································888·····
%  ··························································Y8b·d88P·····
%  ···························································"Y88P"······
%  =======================================================================
% 
%  -----------------------------------------------------------------------
% Author       : 焱铭
% Date         : 2023-12-03 15:43:39 +0800
% LastEditTime : 2024-04-04 23:46:20 +0800
% Github       : https://github.com/YanMing-lxb/
% FilePath     : \GUET_Thesis_LaTeX\Chapters\Chapter5.tex
% Description  : 
%  -----------------------------------------------------------------------
%

% !Mode:: "TeX:UTF-8"

\chapter{算法与定理的示例}\label{ch:5}

\section{算法示例}

\noindent 算法示例如下:

\begin{algorithm}[H]
    \KwData{this text}
    \KwResult{how to write algorithm with \LaTeX2e}
    initialization\;
    \While{not at end of this document}{
        read current\;
        \eIf{understand}{
            go to next section\;
            current section becomes this one\;
        }{
            go back to the beginning of current section\;
        }
    }
    \caption{How to wirte an algorithm.}
    \label{alg:my_algorithm}
\end{algorithm}

算法如\cref{alg:my_algorithm}

\section{定理定义示例}

\begin{theorem}
    XXXXXXXXXX
\end{theorem}
\begin{proof}
    XXXXXXXXXX
\end{proof}
\begin{corollary}
    XXXXXXXXXX
\end{corollary}
\begin{lemma}
    XXXXXXXXXX
\end{lemma}                                              % 可根据需求自行添加章节数

    %
%  =======================================================================
%  ····Y88b···d88P················888b·····d888·d8b·······················
%  ·····Y88b·d88P·················8888b···d8888·Y8P·······················
%  ······Y88o88P··················88888b·d88888···························
%  ·······Y888P··8888b···88888b···888Y88888P888·888·88888b·····d88b·······
%  ········888······"88b·888·"88b·888·Y888P·888·888·888·"88b·d88P"88b·····
%  ········888···d888888·888··888·888··Y8P··888·888·888··888·888··888·····
%  ········888··888··888·888··888·888···"···888·888·888··888·Y88b·888·····
%  ········888··"Y888888·888··888·888·······888·888·888··888··"Y88888·····
%  ·······························································888·····
%  ··························································Y8b·d88P·····
%  ···························································"Y88P"······
%  =======================================================================
% 
%  -----------------------------------------------------------------------
% Author       : 焱铭
% Date         : 2024-05-31 22:39:58 +0800
% LastEditTime : 2024-06-08 16:54:08 +0800
% Github       : https://github.com/YanMing-lxb/
% FilePath     : \GUET_Thesis_LaTeX\Chapters\Conclusion.tex
% Description  : 
%  -----------------------------------------------------------------------
%

% !Mode:: "TeX:UTF-8"

\chapter{全文总结与展望}\label{ch:6}

\section{文字操作}

\hl{高亮显示}:\verb|\hl{}|

\textbf{加粗}:\verb|\textbf{}|

\textit{斜体}:\verb|\textit{}|

\underline{下划线}:\verb|\underline{}|

\uline{下划线}:\verb|\uline{}| 

\uuline{双下划线}:\verb|\uuline{}|

\uwave{波浪线}:\verb|\uwave{}|

\sout{删除线}:\verb|\sout{}|

\xout{斜线}:\verb|\xout{}|

\dotuline{带点的下划线}:\verb|\dotuline{}|

\dashuline{虚线下划线}:\verb|\dashuline{}|


\section{空白符号}
    % 空行分段,多个空行等同于一个
    % 自动缩进,绝对不能使用空行代替
    % 英文中多个空格处理为一个空格,中文中空格会被忽略
    % 汉字与其他字符的间距会自动有XeLaTeX处理
    % 禁止使用中文全角空格
 
    % 1em(当前字体中M的宽度)
    1em: a\quad b
 
    % 2em
    2em: a\qquad b
 
    % 约为1/6个em
    1/6个em: a\,b 或 a\thinspace b
 
    % 0.5个em
    0.5个em: a\enspace b
 
    % 空格
    空格: a\ b
 
    % 硬空格,即不能分割的空格
    硬空格: a~b
 
    % 1pc=12pt=4.218mm
    指定宽度1pc: a\kern 1pc b
 
    指定宽度-1em: a\kern -1em b
 
    指定宽度1em: a\hskip 1em b
 
    指定宽度35pt: a\hspace{35pt}b
 
    % 占位宽度
    占位宽度为xyz: a\hphantom{xyz}b
 
    % 弹性长度hfill命令用于撑满整个空间
    弹性长度: a\hfill b

\section{\LaTeX 控制符}
\#              % 输出井号
\$              % 输出美元符号
\{ \}           % 输出大括号
\~{}            % 输出波浪
\_{}            % 输出下划线
\^{}            % 输出尖角
\textbackslash  %  输出反斜杠
\&              % 输出与符号


\section{后续工作展望}                                            % 总结与展望
    
% ------------------------------------------------------------------------------------------
    \backmatter % 后记部分
    \ThesisBibliography                                                    % 参考文献
    \ThesisAcknowledgement                                                 % 致谢
    \ThesisAchievement                                                     % 攻读专业硕士学位期间取得的成果

    % 如果你想添加多个附录,你可以这样做:

% \ThesisAppendix
% \chapter{附录标题1}

% \section{占位符1}

% 随机文本1

% \chapter{附录标题2}
% 附录内容2

% 如果你只想添加一个附录,你可以这样做:

\ThesisSingleAppendix
% \chapter{附录标题}

\section{占位符2}

随机文本2

\begin{figure}[htb]
    \includegraphics[width=0.8 \textwidth]{50-years-processor-trend.png}
    \caption[处理器发展]{近50年微处理器发展趋势} % 中括号中内容为插图索引中显示内容,可在题注内容过长时使用
    \label{fig:processor-trend}
\end{figure}

图片引用示例:\cref{fig:processor-trend}。
% 在这两个例子中,\chapter命令用于添加一个新的附录章节。你可以在\chapter命令后面写上你的附录内容。                                          % 附录
    
% ------------------------------------------------------------------------------------------    
\end{document} 
