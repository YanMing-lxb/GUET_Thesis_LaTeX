%
%  =======================================================================
%  ····Y88b···d88P················888b·····d888·d8b·······················
%  ·····Y88b·d88P·················8888b···d8888·Y8P·······················
%  ······Y88o88P··················88888b·d88888···························
%  ·······Y888P··8888b···88888b···888Y88888P888·888·88888b·····d88b·······
%  ········888······"88b·888·"88b·888·Y888P·888·888·888·"88b·d88P"88b·····
%  ········888···d888888·888··888·888··Y8P··888·888·888··888·888··888·····
%  ········888··888··888·888··888·888···"···888·888·888··888·Y88b·888·····
%  ········888··"Y888888·888··888·888·······888·888·888··888··"Y88888·····
%  ·······························································888·····
%  ··························································Y8b·d88P·····
%  ···························································"Y88P"······
%  =======================================================================
% 
%  -----------------------------------------------------------------------
% Author       : 焱铭
% Date         : 2023-07-18 08:50:38 +0800
% LastEditTime : 2025-04-21 09:50:01 +0800
% Github       : https://github.com/YanMing-lxb/
% FilePath     : /GUET_Thesis_LaTeX/main.tex
% Description  : 更新请关注 https://github.com/YanMing-lxb/GUET_Thesis_LaTeX
%  -----------------------------------------------------------------------
%

% \special{dvipdfmx:config z 0}                % XeLaTeX取消PDF压缩,加快编译速度,但会增加PDF体积
% \pdfcompresslevel=0                          % PdfLaTeX取消PDF压缩,加快编译速度,但会增加PDF体积
% \pdfobjcompresslevel=0                       % LuaLaTeX取消PDF压缩,加快编译速度,但会增加PDF体积

% ------------------------------------------------------------------------------------------
%     前言区域
% ------------------------------------------------------------------------------------------

\documentclass[master]{GUET-Thesis} 
% pversion:打印版 bversion:盲审版 
% bachelor:本科 master:学硕 promaster:专硕 doctor:博士 ojmaster 在职硕士 ptmaster 非全专硕

% \documentclass[master,bversion]{GUET-Thesis} % 盲审版
% \documentclass[master,pversion]{GUET-Thesis} % 打印版

% ------------------------------------------------------------------------------------------
%     资源路径
% ------------------------------------------------------------------------------------------

\graphicspath{
        {./Pictures/},
        {./Pictures/Chapter1/},
        {./Pictures/Chapter2/},
        {./Pictures/Chapter3/},
        {./Pictures/Chapter4/},
        {./Pictures/Chapter5/}
}                                                % 图片所在位置,根据需求进行修改
\ThesisBibResource{./References/References.bib}  % 参考文献数据源加载
\ThesisAchResource{./References/Accomplishs.bib} % 攻读学位期间取得成果数据源加载

% ------------------------------------------------------------------------------------------
%     封面信息
% ------------------------------------------------------------------------------------------

\Title{                                          % 题目{中文}{英文}  
    基于嵌入式散热模块的微通道散热技术研究
}{
    Research on microchannel heat dissipation technology \\& based on embedded heat dissipation module
    }                                            % 标题中插入“\\&”命令进行换行
\Author{焱铭}                                     % 作者姓名
\Advisor{XXX}                                    % 导师姓名
\Protitle{教授}                                   % 导师职称
\School{机电工程学院}                              % 所在学院
\Major{机械工程}                                  % 学科专业或领域
\ResearchDirection{电子器件散热}                   % 研究领域(盲审用)
\DegreeCategories{工学硕士}                       % 申请学位门类或类别
\StudentNumber{2020XXXXX}                        % 学号
\Secrets{}                                       % 密级,不涉密请空着
\Date{\today}                                    % 可更换为具体日期如:\date{2023年5月28日}
% \subCrefNameSet                                  % 子图、表、算法等引用名称格式设置为 子图 2-1(a) 的样式
\SymbolGroupStyle{符号分类}                       % 符号分组样式,可选参数为“符号分类”或“章节分类”
\theAlgoSet{algorithmicx}                        % 设置算法环境,默认为 algorithmicx,可选参数为 algorithm2e 或 algorithmicx ,注意启用不同宏包时需要适用对应算法命令
% ------------------------------------------------------------------------------------------
%     正文
% ------------------------------------------------------------------------------------------

\begin{document}
    \frontmatter                                 % 前言部分
    \MakeCover                                   % 封面
    \OriginalityDeclaration                      % 独创性声明
    % \SignatureDeclaration{./Chapters/独创性声明(示例).pdf} % 可使用已签字的独创性声明PDF文件
    
    \input{Chapters/Abstract}                    % 摘要

% ------------------------------------------------------------------------------------------ 

    \ThesisFigureList                            % 插图目录
    \ThesisTableList                             % 插表目录
    \ThesisSymbolList                            % 符号说明表
    \ThesisContents                              % 目录

% ------------------------------------------------------------------------------------------
    \mainmatter                                  % 正文部分

    % 对于需要引言并且不需要章节编号且要放在正文中的情况可以使用以下命令
    % \thispagestyle{empty}                      % 保留原有空白页设置
    % \StandardHeader                            % 手动应用标准页眉
    % \chapter*{\texorpdfstring{引\quad 言}{引言}}
    % \addcontentsline{toc}{chapter}{
    %     \texorpdfstring{引\quad 言}{引言}
    % }
    
    \input{Chapters/Symbol}                      % 符号定义文件

    %
%  =======================================================================
%  ····Y88b···d88P················888b·····d888·d8b·······················
%  ·····Y88b·d88P·················8888b···d8888·Y8P·······················
%  ······Y88o88P··················88888b·d88888···························
%  ·······Y888P··8888b···88888b···888Y88888P888·888·88888b·····d88b·······
%  ········888······"88b·888·"88b·888·Y888P·888·888·888·"88b·d88P"88b·····
%  ········888···d888888·888··888·888··Y8P··888·888·888··888·888··888·····
%  ········888··888··888·888··888·888···"···888·888·888··888·Y88b·888·····
%  ········888··"Y888888·888··888·888·······888·888·888··888··"Y88888·····
%  ·······························································888·····
%  ··························································Y8b·d88P·····
%  ···························································"Y88P"······
%  =======================================================================
% 
%  -----------------------------------------------------------------------
% Author       : 焱铭
% Date         : 2024-04-17 20:49:49 +0800
% LastEditTime : 2025-10-26 15:42:27 +0800
% Github       : https://github.com/YanMing-lxb/
% FilePath     : /GUET_Thesis_LaTeX/Chapters/Chapter1.tex
% Description  : 
%  -----------------------------------------------------------------------
%

% !Mode:: "TeX:UTF-8"
%此为第一章节。
%[h]为hear代码所在位置,\caption为表注题注,\cref{}引用图表公式章节等,\cite为引用参考文献,\subfloat子图,\label标签,\begin{figure}图片环境,\begin{table}表格环境,\begin{equation}公式环境,\toprule三线表顶线,\cmidrule三线表中线,\bottomrule三线表底线,\begin{theorem}定理,\begin{proof}证明,\begin{corollary}推论,\begin{lemma}引理
    
\chapter{关于模板的说明}\label{ch:1}


\section{如何看本文档}
本文档简单介绍了模板的一些基础使用方法,在阅读本文档时,应当将代码与PDF文档对照来看,了解各个代码所对应的实现效果。

\section{环境配置与模板参数说明}
请详细阅读本项目根目录下的README.md 文档

\section{节标题示例}

\subsection{小节标题}

\subsubsection{小小节标题}

\section{参考文献插入示例}

    \LaTeX 插入参考文献只需在 \verb|\cite{}| 中输入文献的key即可,如:示例\cite{Lau_2022}。如需插入多个参考文献只需用英文逗号分隔即可,\LaTeX 会自动进行排序整理文献,如:示例\cite{Sadique.Murtaza.ea_2022, Tan.Du.ea_2021, Lau_2022}

    如需参考文献引用样式为平标(非上标样式)只需:\verb|\parencite{}| 中输入文献的key即可,如:示例\parencite{Lau_2022}。如需插入多个参考文献只需用英文逗号分隔即可,\LaTeX 会自动进行排序整理文献,如:示例\parencite{Sadique.Murtaza.ea_2022, Tan.Du.ea_2021, Lau_2022}

    网络资源示例\cite{中华人民共和国公安部_2501}
    
\section{列表示例}

\subsection{纯数字编号}
\begin{enumerate}
 \item XXXXXXXXXX
 \item XXXXXXXXXX
 \item XXXXXXXXXX
\end{enumerate}

\subsection{罗马编号}
\begin{enumerate}[label=(\roman*)]
 \item XXXXXXXXXX
 \item XXXXXXXXXX
 \item XXXXXXXXXX
\end{enumerate}

\subsection{括号编号}
\begin{enumerate}[label=(\arabic*)]
 \item XXXXXXXXXX
 \item XXXXXXXXXX
 \item XXXXXXXXXX
\end{enumerate}

\subsection{半括号编号}
\begin{enumerate}[label=\arabic*)]
 \item XXXXXXXXXX
 \item XXXXXXXXXX
 \item XXXXXXXXXX
\end{enumerate}

\subsection{小字母编号}
\begin{enumerate}[label=\alph*)]
 \item XXXXXXXXXX
 \item XXXXXXXXXX
 \item XXXXXXXXXX
\end{enumerate}

\subsection{自定义编号}
\begin{enumerate}[leftmargin = 6em, labelsep = 0em]
    \item[步骤一、] XXXXXXXXXX;
    \item[步骤二、] XXXXXXXXXX;
    \item[步骤三、] XXXXXXXXXX;
\end{enumerate}

\subsection{无缩进列表}
无缩进列表名为 nienumerate ,意为 noindent enumerate
\begin{nienumerate}
    \item 题目:题目是以最恰当、最简明的词语反映论文中最重要的特定内容的逻辑组合,力求简短切题。中文题目(包括副标题和标点符号)一般不超过20个字,英文题目一般不超过10个实词。
    \item 责任者姓名:包括研究生学号、研究生姓名、指导教师姓名及职称。
    \item 申请学位门类(学硕):按照学科门类和学位层次填写,如工学博士、工学硕士、管理学硕士、经济学硕士等。
    \item 申请学位类别(专硕):按照不同类别分别填写,如工程硕士、法律硕士、翻译硕士、工商管理硕士、会计硕士、艺术硕士。
    \item 领域(专硕):工程硕士、法律硕士、翻译硕士需填写领域,其他几个学位类别无领域,将该行直接删掉。
    \item 论文答辩日期:按照实际答辩日期填写。
\end{nienumerate}

\section{本论文的结构安排}
\cref{ch:1}:绪论。本章主要进行整体说明。

\cref{ch:2}:图片示例。

\cref{ch:3}:表格示例。

\cref{ch:4}:数学公式示例。

\cref{ch:5}:列表、算法、定理、证明插入示例。

\cref{ch:6}:全文总结与展望。本次研究工作进行总结,并根据全文研究过程中……。



    \input{Chapters/Chapter2}
    \input{Chapters/Chapter3}
    %
%  =======================================================================
%  ····Y88b···d88P················888b·····d888·d8b·······················
%  ·····Y88b·d88P·················8888b···d8888·Y8P·······················
%  ······Y88o88P··················88888b·d88888···························
%  ·······Y888P··8888b···88888b···888Y88888P888·888·88888b·····d88b·······
%  ········888······"88b·888·"88b·888·Y888P·888·888·888·"88b·d88P"88b·····
%  ········888···d888888·888··888·888··Y8P··888·888·888··888·888··888·····
%  ········888··888··888·888··888·888···"···888·888·888··888·Y88b·888·····
%  ········888··"Y888888·888··888·888·······888·888·888··888··"Y88888·····
%  ·······························································888·····
%  ··························································Y8b·d88P·····
%  ···························································"Y88P"······
%  =======================================================================
% 
%  -----------------------------------------------------------------------
% Author       : 焱铭
% Date         : 2024-05-31 22:39:58 +0800
% LastEditTime : 2025-02-07 14:27:36 +0800
% Github       : https://github.com/YanMing-lxb/
% FilePath     : /GUET_Thesis_LaTeX/Chapters/Chapter4.tex
% Description  : 
%  -----------------------------------------------------------------------
%

% !Mode:: "TeX:UTF-8"

\chapter{数学公式示例}\label{ch:4}

本章在\cref{ch:3}完成基……。

\section{公式示例}
在本次研究中应用到计算流体动力学(Computational Fluid Dynamics,CFD)对研究对象进……。


\subsection{普通带序号公式}
\begin{equation}
    \frac{\partial u}{\partial x}+\frac{\partial v}{\partial y}+\frac{\partial v}{\partial z}=0
\end{equation}
u,v,w 分别是 x,y,z 方向的速度分量。


\subsection{需要对齐的多个带序号的公式}

\& 号为对其对齐标记最好放置在计算符号之前,如=、+、-之前;

\verb|\\| 表示换行。

\begin{align}% 式中的&为对齐的位置标记
    u & \frac{\partial u}{\partial x}+v \frac{\partial u}{\partial y}+w \frac{\partial u}{\partial z}=-\frac{1}{\rho_{f}} \frac{\partial p}{\partial x}+\frac{\mu_{f}}{\rho_{f}}\left(\frac{\partial^{2} u}{\partial x^{2}}+\frac{\partial^{2} u}{\partial y^{2}}+\frac{\partial^{2} u}{\partial z^{2}}\right) \\
    u & \frac{\partial v}{\partial x}+v \frac{\partial v}{\partial y}+w \frac{\partial v}{\partial z}=-\frac{1}{\rho_{f}} \frac{\partial p}{\partial y}+\frac{\mu_{f}}{\rho_{f}}\left(\frac{\partial^{2} v}{\partial x^{2}}+\frac{\partial^{2} v}{\partial y^{3}}+\frac{\partial^{2} v}{\partial z^{3}}\right) \\
    u & \frac{\partial w}{\partial x}+v \frac{\partial w}{\partial y}+w \frac{\partial w}{\partial z}=-\frac{1}{\rho_{f}} \frac{\partial p}{\partial z}+\frac{\mu_{f}}{\rho_{f}}\left(\frac{\partial^{2} w}{\partial x^{2}}+\frac{\partial^{2} w}{\partial y^{2}}+\frac{\partial^{2} w}{\partial z^{2}}\right)
\end{align}
$\rho_{f}$ 和 $\mu_{f}$ 分别是冷却剂的密度和动态粘度,p 是冷却剂压力。


\subsection{需要换行对齐的长公式}

\begin{equation}\label{eq:P}
    \begin{split}
        f^{\ \prime} & = 6.272 + 3.02 A + 6.08 B + 0.0368 C - 0.8848 D  \\
            & + 0.04381 D^2 + 6.35 AB - 0.3602 AD - 0.5497 BD
    \end{split}
\end{equation}

\subsection{其他公式示例}
\begin{equation}
    \begin{aligned}
    \left\{
        \begin{array}{l}
        \text {find}\enspace H_{rib},H_{rib},N_{pf},N_{ac} \\
        \text {min} \enspace F(H_{rib},H_{rib},N_{pf},N_{ac})= min\{f_1,f_2,f_3\} \\

            \text{s.t.\enspace}\enspace 0.2 \leqslant H_{rib} \leqslant 0.8    \\
            \hspace{2.2em} 0.2 \leqslant H_{pf} \leqslant 0.8                     \\
            \hspace{2.2em} 6 \leqslant N_{pf} \leqslant 22,\ N_{pf}\in \mathbb{O} \\
            \hspace{2.2em} 0 \leqslant N_{ac} \leqslant 16,\ N_{ac}\in \mathbb{E}
        \end{array}
    \right. 
    \end{aligned}
    \label{eq:MO}
\end{equation}

\section{本章小节}
    \input{Chapters/Chapter5}                    % 可根据需求自行添加章节数

    \input{Chapters/Conclusion}                  % 总结与展望
    
% ------------------------------------------------------------------------------------------
    \backmatter                                  % 后记部分
    \ThesisBibliography                          % 参考文献
    \ThesisAcknowledgement                       % 致谢
    \ThesisAchievement                           % 攻读专业硕士学位期间取得的成果

    % 如果你想添加多个附录,你可以这样做:

% \ThesisAppendix
% \chapter{附录标题1}

% \section{占位符1}

% 随机文本1

% \chapter{附录标题2}
% 附录内容2

% 如果你只想添加一个附录,你可以这样做:

\ThesisSingleAppendix
% \chapter{附录标题}

\section{占位符2}

随机文本2

\begin{figure}[htb]
    \includegraphics[width=0.8 \textwidth]{50-years-processor-trend.png}
    \caption[处理器发展]{近50年微处理器发展趋势} % 中括号中内容为插图索引中显示内容,可在题注内容过长时使用
    \label{fig:processor1}
\end{figure}

图片引用示例:\cref{fig:processor-trend}。
% 在这两个例子中,\chapter命令用于添加一个新的附录章节。你可以在\chapter命令后面写上你的附录内容。                % 附录
    
% ------------------------------------------------------------------------------------------    
\end{document} 
