%
%  =======================================================================
%  ····Y88b···d88P················888b·····d888·d8b·······················
%  ·····Y88b·d88P·················8888b···d8888·Y8P·······················
%  ······Y88o88P··················88888b·d88888···························
%  ·······Y888P··8888b···88888b···888Y88888P888·888·88888b·····d88b·······
%  ········888······"88b·888·"88b·888·Y888P·888·888·888·"88b·d88P"88b·····
%  ········888···d888888·888··888·888··Y8P··888·888·888··888·888··888·····
%  ········888··888··888·888··888·888···"···888·888·888··888·Y88b·888·····
%  ········888··"Y888888·888··888·888·······888·888·888··888··"Y88888·····
%  ·······························································888·····
%  ··························································Y8b·d88P·····
%  ···························································"Y88P"······
%  =======================================================================
% 
%  -----------------------------------------------------------------------
% Author       : 焱铭
% Date         : 2024-05-31 22:39:58 +0800
% LastEditTime : 2024-06-08 16:54:08 +0800
% Github       : https://github.com/YanMing-lxb/
% FilePath     : \GUET_Thesis_LaTeX\Chapters\Conclusion.tex
% Description  : 
%  -----------------------------------------------------------------------
%

% !Mode:: "TeX:UTF-8"

\chapter{全文总结与展望}\label{ch:6}

\section{文字操作}

\hl{高亮显示}:\verb|\hl{}|

\textbf{加粗}:\verb|\textbf{}|

\textit{斜体}:\verb|\textit{}|

\underline{下划线}:\verb|\underline{}|

\uline{下划线}:\verb|\uline{}| 

\uuline{双下划线}:\verb|\uuline{}|

\uwave{波浪线}:\verb|\uwave{}|

\sout{删除线}:\verb|\sout{}|

\xout{斜线}:\verb|\xout{}|

\dotuline{带点的下划线}:\verb|\dotuline{}|

\dashuline{虚线下划线}:\verb|\dashuline{}|


\section{空白符号}
    % 空行分段,多个空行等同于一个
    % 自动缩进,绝对不能使用空行代替
    % 英文中多个空格处理为一个空格,中文中空格会被忽略
    % 汉字与其他字符的间距会自动有XeLaTeX处理
    % 禁止使用中文全角空格
 
    % 1em(当前字体中M的宽度)
    1em: a\quad b
 
    % 2em
    2em: a\qquad b
 
    % 约为1/6个em
    1/6个em: a\,b 或 a\thinspace b
 
    % 0.5个em
    0.5个em: a\enspace b
 
    % 空格
    空格: a\ b
 
    % 硬空格,即不能分割的空格
    硬空格: a~b
 
    % 1pc=12pt=4.218mm
    指定宽度1pc: a\kern 1pc b
 
    指定宽度-1em: a\kern -1em b
 
    指定宽度1em: a\hskip 1em b
 
    指定宽度35pt: a\hspace{35pt}b
 
    % 占位宽度
    占位宽度为xyz: a\hphantom{xyz}b
 
    % 弹性长度hfill命令用于撑满整个空间
    弹性长度: a\hfill b

\section{\LaTeX 控制符}
\#              % 输出井号
\$              % 输出美元符号
\{ \}           % 输出大括号
\~{}            % 输出波浪
\_{}            % 输出下划线
\^{}            % 输出尖角
\textbackslash  %  输出反斜杠
\&              % 输出与符号


\section{后续工作展望}